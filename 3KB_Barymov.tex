\documentclass[12pt,a4paper]{article}
\usepackage{amsmath,amssymb,amsthm}
\usepackage{algorithm}
\usepackage[noend]{algpseudocode} 
\usepackage[utf8]{inputenc}
\usepackage[russian]{babel}
\usepackage{geometry}
\usepackage{listings}
\usepackage{xcolor}
\definecolor{codegreen}{rgb}{0,0.6,0}
\definecolor{codegray}{rgb}{0.5,0.5,0.5}
\definecolor{codepurple}{rgb}{0.58,0,0.82}

\lstset{
    basicstyle=\ttfamily\small,       
    keywordstyle=\color{blue},        
    commentstyle=\color{green},       
    numbers=left,                     
    frame=single,                     
    breaklines=true,                  
    showstringspaces=false,           
    tabsize=4,                        
    numberstyle=\tiny\color{gray},    
    captionpos=t                      
}

\lstset{style=mystyle}
\geometry{
    a4paper,
    left=30mm,
    right=15mm,
    top=20mm,
    bottom=20mm
}

% PROBABILITY SYMBOLS
\newcommand*\PROB\Pr 
\DeclareMathOperator*{\EXPECT}{\mathbb{E}}

% Sets, Rngs, ets 
\newcommand{\N}{{{\mathbb N}}}
\newcommand{\Z}{{{\mathbb Z}}}
\newcommand{\R}{{{\mathbb R}}}
\newcommand{\Zp}{\ints_p} % Integers modulo p
\newcommand{\Zq}{\ints_q} % Integers modulo q
\newcommand{\Zn}{\ints_N} % Integers modulo N

% Landau 
\newcommand{\bigO}{\mathcal{O}}
\newcommand*{\OLandau}{\bigO}
\newcommand*{\WLandau}{\Omega}
\newcommand*{\xOLandau}{\widetilde{\OLandau}}
\newcommand*{\xWLandau}{\widetilde{\WLandau}}
\newcommand*{\TLandau}{\Theta}
\newcommand*{\xTLandau}{\widetilde{\TLandau}}
\newcommand{\smallo}{o} %technically, an omicron
\newcommand{\softO}{\widetilde{\bigO}}
\newcommand{\wLandau}{\omega}
\newcommand{\negl}{\mathrm{negl}} 

% Misc
\newcommand{\eps}{\varepsilon}
\newcommand{\inprod}[1]{\left\langle #1 \right\rangle}

\newcommand{\handout}[5]{
  \noindent
  \begin{center}
  \framebox{
    \vbox{
      \hbox to 5.78in { {\bf Учебная практика (учебно-
лабораторная)} \hfill #2 }
      \vspace{4mm}
      \hbox to 5.78in { {\Large \hfill #5  \hfill} }
      \vspace{2mm}
      \hbox to 5.78in { {\em #3 \hfill #4} }
    }
  }
  \end{center}
  \vspace*{4mm}
}

\newcommand{\lecture}[4]{\handout{#1}{#2}{#3}{Автор: #4}{Название темы #1}}

\newtheorem{theorem}{Теорема}
\newtheorem{lemma}{Лемма}
\newtheorem{definition}{Определение}
\newtheorem{corollary}{Следствие}
\newtheorem{fact}{Факт}

\begin{document}

\begin{titlepage}
    \centering
    % Название министерства
    {\large\textbf{Федеральное государственное автономное образовательное учреждение высшего образования}}\par
    \vspace{0.5cm}
    % Название университета
    {\large\textbf{Балтийский федеральный университет имени Иммануила Канта}}\par
    \vspace{0.5cm}
    
    % Институт
    {\large Высшая школа компьютерных наук и искусственного интеллекта}\par
    \vspace{3cm}
    
    % Тип работы
    {\Large\textbf{ОТЧЕТ}}\par
    \vspace{1cm}
    
    % Название работы
    {\Large по учебной практике (учебно-лабораторной)}\par
    \vspace{2cm}
    
    % Информация о студенте (выровнено по правому краю)
    \begin{flushright}
        \large
        \textbf{Выполнил:}\par
        студент \underline{Барымов И.А.}\par
        \vspace{0.5cm}
        группы \underline{05-КБ-22-О}\par
        \vspace{0.5cm}
        направление подготовки 10.05.01 <<Компьютерная безопасность>>\par
        \vspace{2cm}
        
        \textbf{Руководитель практики:}\par
        ст. преподаватель \underline{Глинчиков К.Е.}\par
        \vspace{2cm}
    \end{flushright}
    
    % Подписи и дата 
    \begin{flushright}
        Работа выполнена:  <<\underline{\hspace{1cm}}>> \underline{\hspace{3cm}} 2025 г.\par
        \vspace{1cm}
        Работа защищена с оценкой: \underline{\hspace{6cm}}\par
        \vspace{1cm}
        <<\underline{\hspace{1cm}}>> \underline{\hspace{3cm}} 2025 г.\par
    \end{flushright}
    
    % Город и год
    \vspace{2cm}
    {\large Калининград 2025}
\end{titlepage}

\newpage

\lecture{Алгоритм Барретта}{6 семестр 2025 г.}{}{Барымов И.А.}
\section{Введение}
\textbf{Алгоритм Барретта} - это алгоритм приведения, предназначенный для оптимизации вычисления выражения x mod N без использования быстрого алгоритма деления, заменяющий операции деления на умножения.

\section{Алгоритм Барретта}

\begin{algorithm}[ph]
	\caption{Классическая модульная редукция по Барретту}
	\label{alg:BarrettReduction}
	\textbf{Вход:} $x$ - число для редукции , $N$ - модуль, \\
	\quad \quad $\mu = \left\lfloor \frac{2^k}{N} \right\rfloor$ (предвычислено), $k = \lfloor \log_2 N \rfloor + 1$ \\
	\textbf{Выход:} $x \mod N$
	
	\begin{algorithmic}[1]
		\State $q \leftarrow \left\lfloor \frac{x \cdot \mu}{2^k} \right\rfloor$ \Comment{Приближение частного}
		\State $r \leftarrow x - q \cdot N$ \Comment{Промежуточный остаток}
		\While{$r \geq N$} \Comment{Корректировка}
		    \State $r \leftarrow r - N$
		\EndWhile
		\State \Return $r$
	\end{algorithmic}
\end{algorithm}

\section{Теоретическое обоснование}

Для алгоритма верна следующая цепочка вычислений:
\begin{enumerate}
    \item \textbf{Константа:} 
    \[\mu = \left\lfloor \frac{2^k}{N} \right\rfloor, \quad \text{где } k = \lfloor \log_2 N \rfloor + 1.\]
    
    \item \textbf{Частное:} 
    \[q = \left\lfloor \frac{x \cdot \mu}{2^k} \right\rfloor.\]
    
    \item \textbf{Оценка:} 
    \[q \leq \frac{x \cdot \mu}{2^k} \leq \frac{x \cdot (2^k / N)}{2^k} = \frac{x}{N}.\] 
    
    \item \textbf{Остаток:}
    \[r = x - q \cdot N\ , \quad \text{где } r < 2N\]
\end{enumerate}

\begin{fact}
Алгоритм возвращает корректный остаток $r = x \mod N$, причём:
\begin{itemize}
    \item Промежуточное значение $r$ удовлетворяет $0 \leq r < 2N$
\end{itemize}
\end{fact}

\subsection{Доказательство}
Из оценки для $q$ следует:
\[ x - qN \geq x - \left(\frac{x}{N}\right)N = 0, \]
а верхняя граница:
\[ x - qN \leq x - \left(\frac{x \cdot \mu}{2^k} - 1\right)N < N + \frac{x}{2^k} \cdot (2^k - \mu N). \]
Поскольку $\mu \geq \frac{2^k}{N} - 1$, то $r < 2N$.
\subsection{Замечания}
\begin{itemize}
    \item Алгоритм особенно эффективен при многократных редукциях по одному модулю (например, в криптографии), так как $\mu$ вычисляется один раз.
    \item Основное преимущество — замена деления на умножение и битовые операции, что намного быстрее на большинстве процессоров.
\end{itemize}

\section{Пример вычисления по алгоритму Барретта}

Пусть:
\begin{itemize}
    \item Модуль $N = 17$
    \item Число для редукции $x = 245$
    \item Параметр $k = 5$ (так как $2^4 \leq N < 2^5$)
\end{itemize}

\subsection{Шаг 1: Предвычисление константы $\mu$}
\[
\mu = \left\lfloor \frac{2^{2k}}{N} \right\rfloor = \left\lfloor \frac{2^{10}}{17} \right\rfloor = \left\lfloor \frac{1024}{17} \right\rfloor = 60
\]

\subsection{Шаг 2: Вычисление приближённого частного $q$}
\[
q = \left\lfloor \frac{x \cdot \mu}{2^{2k}} \right\rfloor = \left\lfloor \frac{245 \times 60}{1024} \right\rfloor = \left\lfloor \frac{14700}{1024} \right\rfloor = 14
\]

\subsection{Шаг 3: Вычисление остатка $r$}
\[
r = x - q \cdot N = 245 - 14 \times 17 = 245 - 238 = 7
\]

\subsection{Шаг 4: Корректировка (ситуативно)}
Так как $7 < 17$, коррекция не выполняется.

\subsection{Результат}
\[
245 \mod 17 = 7 \quad \text{(верно, так как $17 \times 14 + 7 = 245$)}
\]

\section{Сравнение и тесты алгоритма со встроенной функцией Sage}


Для сравнения эффективности реализации алгоритма Барретта со встроенной операцией модуля был проведён ряд тестов. Результаты представлены в таблице~\ref{tab:speed_comparison}.

\begin{table}[h]
\centering
\caption{Сравнение времени работы алгоритмов (в секундах)}
\label{tab:speed_comparison}
\begin{tabular}{|c|c|c|c|}
\hline
\textbf{Размерность $N$} & \textbf{Алгоритм Барретта} & \textbf{Встроенная операция \%} & \textbf{Отношение} \\
\hline
110 бит & 0.000659 & 0.000183 & 0.000475 \\
115 бит & 0.002742 & 0.000134 & 0.002607 \\
120 бит & 0.158559 & 0.000117 & 0.158442 \\
125 бит & 3.063338 & 0.000080 & 3.063258 \\
\hline
\end{tabular}
\end{table}

\begin{itemize}
\item Все тесты проводились на процессоре Intel Core i3-10100F
\item Время приведено в секундах с точностью до 6 знаков после запятой
\end{itemize}

\subsection{Методика тестирования}
Тестирование проводилось на следующей конфигурации:
\begin{itemize}
\item Процессор: Intel Core i3-10100F (3.6 GHz)
\item Память: 16 ГБ DDR4
\item ОС: Windows 10
\item SageMathCell - десктопная версия браузерной ячейки
\end{itemize}
\newpage
\subsection{Анализ результатов}

Как видно из таблицы~\ref{tab:speed_comparison}:
\begin{itemize}
\item Встроенная функция SageMath работает сильно быстрее алгоритма Барретта(в силу оптимизации Sage)
\item Разница в скорости растёт при увеличении размера входных данных
\end{itemize}

\begin{lstlisting}[language=Python,caption=Код для тестирования]
import time

def barrett_reduce(x, N, mu, k):   
    q = (x * mu) >> k  
    r = x - q * N
    
    while r >= N:
        r -= N
    return r

N = 2^50-1
x = 2^125
k = N.nbits()  
mu = (1 << k) // N  

start = time.time()
res1 = x % N
t1 = time.time() - start

start = time.time()
res2 = barrett_reduce(x, N, mu, k)
t2 = time.time() - start
if (res1==res2):
    print(f"%: {t1:.6f} sec")
    print(f"Barrett:{t2:.6f} sec")
\end{lstlisting}

\section{Применение в криптографии}

Алгоритм Барретта применяется в:

\begin{itemize}
\item Модульном возведении в степень в RSA и Диффи-Хеллмане
\item Эллиптической криптографии (ECC)
\item Постквантовой криптографии 
\end{itemize}

\newpage
\begin{center}
\paragraph{Список литературы}
\end{center}

\begin{enumerate}
\item Barrett P. Implementing the Rivest Shamir and Adleman Public Key Encryption Algorithm on a Standard Digital Signal Processor // Advances in Cryptology - CRYPTO'86. - 1987. - P. 311-326.
\item Панкратова И. А. - Теоретико-числовые методы в криптографии: учебное пособие. — 2009. — С. 12-13
\end{enumerate}

\end{document}